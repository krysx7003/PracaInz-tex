\section*{Streszczenie}

Celem poniższej pracy było wykonanie autorskiego zbioru danych, który mógłby posłużyć
do testowania wydajności algorytmów optycznego rozpoznawania znaków.
Zbiór został wykonany przy pomocy interfejsu programistycznego WolneLektury.pl, Co 
zapewniło dostęp do bogatego zbioru tekstów literackich w języku polskim, dostępnych 
w domenie publicznej.

Korpus zawiera teksty wykonane w czterech różnorodnych krojach pisma, w tym takich
, które przypominają pismo odręczne. Ponadto korpus zawiera obrazy poddane syntetycznym
zniekształceniom wizualnym, które mają symulować rzeczywistą degradację dokumentów.

Tak przygotowany zbiór danych wykorzystano do ewaluacji czterech różnych algorytmów
optycznego rozpoznawania znaków (OCR): DocTR, EasyOCR, Paddle oraz Tesseract. 
Analiza porównawcza objęła zarówno metryki dokładności (CER), jak i wydajności
czasowej przetwarzania.

Wyniki badań wskazują na statystycznie istotne różnice w efektywności wybranych 
algorytmów w zależności od zastosowanego kroju pisma oraz rodzaju zniekształcenia.
Opracowany zbiór danych wykazał się wartością diagnostyczną, umożliwiając identyfikację
mocnych i słabych stron badanych implementacji OCR.

\section*{Abstract}

The goal of this work was the creation of a dataset. Its main use case would be 
for benchmarking optical character recognition algorithms. The dataset was 
made using the WolneLektury.pl API. Doing so provided access to a rich set of literature
in the Polish language, which formed a great basis for the creation of the dataset. All
books used in the creation of this dataset are in the public domain.

This set contains texts made using four different fonts, two of which are designed
to imitate handwriting. Additionally, the dataset contains images affected by 
synthetic distortions, which should simulate realistic degradation of document
quality.

Next, the created dataset was used to evaluate four different OCRs: DocTR, EasyOCR,
Paddle, and Tesseract. Results were analyzed by comparing both their accuracy (CER)
and runtime.

Results of the experimentation show statistically significant differences in the performances
of OCRs depending on the font and type of distortion. The dataset shows promise for diagnosing,
benchmarking, and identifying the strengths and weaknesses of OCRs.
