\section*{Streszczenie}

Celem poniższej pracy było wykonanie autorskiego zbioru danych, który mógłby posłużyć
do testowania wydajności algorytmów optycznego rozpoznawania znaków.
Zbiór został wykonany przy pomocy interfejsu programistycznego WolneLektury.pl. Co 
zapewniło dostęp do bogatego zbioru tekstów literackich w języku polskim, dostępnych 
w domenie publicznej.

Korpus zawiera teksty wykonane w czterech różnorodnych krojach pisma, w tym takich
przypominających pismo odręczne. Ponadto korpus zawiera obrazy poddane syntetycznym
zniekształceniom wizualnym, które mają symulować rzeczywistą degradację dokumentów.

Tak przygotowany zbiór danych wykorzystano do ewaluacji czterech różnych algorytmów
optycznego rozpoznawania znaków (OCR): DocTR, EasyOCR, Paddle oraz Tesseract. 
Analiza porównawcza objęła zarówno metryki dokładności (CER), jak i wydajności
czasowej przetwarzania.

Wyniki badań wskazują na statystycznie istone różnice w efektywnośći wybranych 
algorytmów w zależności od zastosowanego kroju pisma oraz rodzaju zniekształcenia.
Opracowany zbiór danych wykazał się wartością diagnostyczną, umożliwiając identyfikację
mocnych i słabych stron badanych implementacji OCR.

\section*{Abstract}

The goal of this work was creation of dataset. Its main use case would be 
for benchmarking optical character recognition algorithms. Said dataset was 
made using WolneLektury.pl API. Doing so, gave acces to rich set of literature
in Polish language, that made great basis for creation of said dataset. All
books used in creation of this dataset are in public domain.

This set contains texts made using four different fonts, two of which are made
to imitate handwirting. Additionally dataset contains images affected by 
synthetic distortions, which should simulate realistic degradation of documents
quality.

Next created dataset was used in evaluating four different OCRs: DocTR, EasyOCR,
Paddle oraz Tesseract. Results were analysed by comparing both their acuracy (CER)
and runtime.

Results of experimentation show statistically significant differences in performances
of OCRs depending on font and type of distortion. Dataset shows promise for diagnosing,
benchmarking and identifiying strong and weak sides of OCRs.
