\Chapter{Wstęp}\label{chapter:wstep}

W wyniku skanowania dokumentów powstają obrazy zawierające teksty. Jednakże większość narzędzi do przetwarzania tekstu
nie jest w stanie operować na plikach graficznych. Aby rozwiązać ten problem, można zastosować algorytmy optycznego
rozpoznawania znaków (Optical Character Recognition lub OCR). Są to narzędzia, które umożliwiają rozpoznanie tekstu na
obrazie oraz zapisanie go w formie bardziej dogodnej do przetwarzania.

Jeden z pierwszych takich algorytmów powstał w roku 1974, a jego twórcą był Ray Kurzweil. Początkowo algorytm ten miał
na celu ułatwienie funkcjonowania osobom z niepełnosprawnością wzrokową. Tekst był najpierw skanowany, a następnie 
przetwarzany przez OCR, dzięki czemu mógł on zostać odczytany na głos. Pierwsze algorytmy rozpoznawania znaków
opierały się na skomplikowanym zbiorze zasad i reguł. Wraz ze wzrostem popularności sztucznej inteligencji oraz metod
uczenia maszynowego, algorytmy OCR stopniowo zaczęły korzystać z coraz częściej korzystać z tych technologii. Dzięki 
temu nie tylko poprawiła się dokładność tych algorytmów, ale także zwiększyła się liczba obsługiwanych krojów pisma 
oraz języków. Obecnie zdecydowana większość takich algorytmów wykorzystuje zaawansowane metody sztucznej inteligencji.

Współcześnie narzędzia te nadal służą osobom z różnymi niepełnosprawnościami, jednakże zakres zastosowań algorytmów OCR
znacznie się poszerzył i obejmuje inne dziedziny. Są one niezbędnym elementem w procesie digitalizacji
tekstów historycznych, automatyzacji procesów biznesowych, przetwarzania dokumentów urzędowych oraz w aplikacjach mobilnych.

Mimo tego że algorytmy OCR są bardzo popularne, jest kilka kluczowych problemów, które mogą wpłynąć na ich funkcjonowanie. 
Jednym z nich jest pismo odręczne (oraz kroje pisma je imitujące). Ten typ pisma jest bardzo zróżnicowany każdy człowiek 
ma swój własny charakter pisma. Zapisane znaki różnią się kształtem, rozmiarem czy nachyleniem. Ponad to zdarza się, że
w tym samym zdaniu te same znaki mogą się znacznie różnić. Wszystkie te czynniki sprawiają, że wiele algorytmów OCR
ma mniejszą skuteczność w rozpoznawaniu pisma odręcznego od pisma maszynowego.

Kolejnym problemem jest jakość dostarczonego obrazu. W rzeczywistości rzadko spotykane są przypadki gdy, skanowany tekst
jest równomiernie oświetlony, nie ma na nim cieni lub papier, na którym znajduje się tekst, jest niepogięty. Takie 
zniekształcenia mogą utrudnić rozdzielenie tekstu od tła, przez co skuteczność algorytmu zmaleje.

\section{Cel pracy}
Celem pracy było utworzenie autorskiego zbioru danych służącego do testowania wydajności algorytmów OCR. Zbiór zawiera
dokumenty o zróżnicowanej charakterystyce, uwzględniając m.in. różne kroje, a także modyfikacje utrudniające poprawne 
odczytanie treści. Zbiór ten został opracowany przy pomocy interfejsu programistycznego (API) serwisu WolneLektury.pl.
Serwis ten zawiera bogaty zbiór książek, opowiadań oraz wierszy, wszystkie dostępne tam pozycje znajdują się w domenie
publicznej.

