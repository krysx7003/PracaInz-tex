\label{sec:tesseract}
Tesseract OCR to najstarszy z wybranych algorytmów optycznego rozpoznawania znaków.
Został on stworzony przez firmę HP w latach 1984 - 1994.
Obecnie Tesseract jest jednym z najpopularniejszych algorytmów OCR w kręgach akademickich\cite{ayyasamy2015,li2023,chowdhury2025}
Do badań użyto wersji 5.3.4 jest to znaczące gdyż wcześniejsze wersje algorytmu(do wersji 3 włącznie) nie wykorzystywały sieci neuronowych.
Algorytm ten działa w następujących krokach:
\begin{enumerate}
    \item Analiza komponentów, gdzie zarys tych komponentów jest przechowywany. 
        Takie podejście mimo że nakłada dodatkowe koszty obliczeniowe pozwala na łatwiejsze rozpoznawanie 
        tekstu w odwróconych kolorach (biały tekst na czarnym tle)\cite{smith2007}.
    \item Wyszukiwanie linii w komponentach. 
        Celem tego kroku była eliminacja potrzeby korekty przekrzywienia.
    \item Podział linii na słowa.
    \item Pierwsza iteracja rozpoznawania. 
        Zaczynając na górze strony algorytm próbuje rozpoznać każde kolejne słowo. Tokeny rozpoznane z wysokim stopniem
        pewności są wykorzystywane do douczenia klasyfikatora. W ten sposób z każdym kolejnym słowem
        celność klasyfikatora powinna rosnąć.
    \item Druga iteracja rozpoznawania.
        Po wykonaniu pierwszej iteracji istnieje szanasa, że klasyfikator uzyskałby lepsze wyniki. Więc
        po raz drugi algorytm próbuje rozpoznać tekst na stronie i aktualizuje słowa które były mniej celnie rozpoznane.

\end{enumerate}


