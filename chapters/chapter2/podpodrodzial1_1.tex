Tesseract OCR to najstarszy z wybranych algorytmów optycznego rozpoznawania znaków.
Został on stworzony przez firmę HP w latach 1984 - 1994.
Algorytm ten działa w kilku fazach:
\begin{enumerate}
    \item Analiza komponentów, gdzie zarys tych komponentów jest przechowywany. 
        Takie podejście mimo że nakłada dodatkowe koszty obliczeniowe pozwala na łatwiejsze rozpoznawanie 
        tekstu w odwróconych kolorach (biały tekst na czarnym tle)\cite{smith2007}.
    \item Wyszukiwanie lini w komponentach. 
        Celem tego kroku była eliminacja potrzeby korekty przekrzywienia.
    \item Podział linii na słowa.
    \item Pierwsza iteracja rozpoznawania. 
        Zaczynając na górze strony algorytm próbuje rozpoznać każde kolejne słowo. Jeżeli jest duże prawdopodobieństwo
        że słowo jest poprawne jest ono wykorzystywane do douczenia klasyfikatora. W ten sposób z każdym kolejnym słowem
        celność klasyfikatora powinna rosnąć.
    \item Druga iteracja rozpoznawania.
        Po wykonaniu pierwszej iteracji jest duże prawdopodobieństow że klasyfikator uzyskałby lepsze wyniki. Więc
        po raz drugi algorytm próbuje rozpoznać tekst na stronie i aktualizuje słowa które były minej celnie rozpoznane.

\end{enumerate}
