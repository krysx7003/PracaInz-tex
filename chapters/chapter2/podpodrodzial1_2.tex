\label{sec:paddle}
Pomimo tego, że algorytm ten jest stosunkowo nowy (pierwsze wersje zostały wypuszczone w 2020 roku\cite{cui2025paddleocr}).
Paddle OCR jest drugim pod względem popularności algorytmem(zaraz po Tesserakcie).
Początkowo pierwsze wersje algorytmu skupiały się na balansie między jakością wyniku a czasem jego otrzymania.
Wraz z czasem w kolejnych wersjach udoskonalano wydajność algorytmu oraz rozszerzano jego umiejętności 
(Np. obsługą wielu języków, rozpoznawanie pisma ręcznego).
Najnowsze wersje tego algorytmu(W pracy użyto wersji 3.2.0) składają się trzech głównych modułów.
\textbf{PP-OCR} Rdzeń całego algorytmu służący do rozpoznawania znaków na obrazie.
\textbf{PP-Structure} Moduł służący do rozpoznawania u strukturyzowanych obrazów (np. Zawierających tabele).
\textbf{PP-ChatOCR} Moduł służący do ekstrakcji kluczowych informacji z obrazów przy pomocy dużego modelu językowego(z ang. Larg Language Model lub LLM).
W tej pracy zastosowany został jedynie moduł PP-OCR.
Działa on w następujących krokach:
\begin{enumerate}
    \item \textbf{Preprocesowanie} - W celu uzyskania jak najlepszej jakości obrazu algorytm może usunąć niektóre zniekształcenia oraz problemy z orientacją obrazu. 
    \item \textbf{Wykrycie tekstu} - Algorytm tworzy mapę prawdopodobieństwa, w której każdy piksel ma przydzieloną wartość określającą jakie jest prawdopodobieństwo,
        że piksel ten jest częścią obszaru tekstowego. Następnie przy pomocy binaryzacji różniczkowalnej (ang. Differentiable Binarization) dynamicznie
        określany jest próg pomiędzy tekstem a tłem. Ostatecznie na podstawie tej mapy tworzone są wielokąty będące zarysem obszaru tekstowego.
    \item \textbf{Wykrycie orientacji linii} - Wykryty tekst dzielony jest na linie. Następnie algorytm upewnia się, że wykryte linie tekstu są w prawidłowej orientacji 
    \item \textbf{Rozpoznanie tekstu} - Poprzez zastosowanie konwolucyjnej sieci neuronowej (z ang. Convolutional Neural Network lub CNN)
        wykrywane są charakterystyczne elementy tekstu jak pociągnięcia, krzywe i pętle. Elementy te podawane są do rekurencyjnej sieci neuronowej 
        (z ang. Recurrent Neural Network lub RNN), która dzięki możliwości "zapamiętania" poprzednich elementów jest w stanie odróżnić poszczególne znaki.
        Na koniec koneksjonistyczna klasyfikacja czasowa (z ang. Connectionist Temporal Classification lub CTC) działa jako mechanizm wyrównywania i zwijania 
        i znajduje najbardziej prawdopodobny napis (Np. "ccccczzzzzaaass" zostaje zmienione na "czas") 
\end{enumerate}
