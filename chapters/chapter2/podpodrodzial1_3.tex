\label{sec:docTR}
Algorytm ten jest skupiony na rozpoznawaniu dokumentów takich jak skany faktur, paragonów, formularzy czy listów.
Stąd też nazwa Document Text Recognition czy docTR w skrócie.  
Główną filozofią tego projektu jest "bezproblemowe optyczne rozpoznawanie znaków dostępne dla każdego"\cite{doctr2021}.
Algorytm ten stosuje dwuetapowe podejście do rozpoznawania tekstu:
\begin{enumerate}
    \item \textbf{Wkrywanie tekstu} - DocTr pozwala na wykorzystanie w tym celu wielu różnych modeli.
        Jednakże większość z nich,podobnie jak w \nameref{sec:tesseract} i \nameref{sec:paddle}, oparta jest na konwolucyjnych 
        sieciach neuronowych. Jednakże w przeciwieństwie do tych dwóch algorytmów docTr używa piramidy cech (z ang. Feature 
        Pyramid Network) co pozwala na odczyt tekstu w wielu różnych rozmiarach (Przydatne na przykład do odróżnienia nagłówka od 
        adnotacji itp.). Następnie algorytm dla każdego piksela w obrazie przewiduje czy jest on w obszarze tekstowym i tworzy 
        wielokąty wokół wykrytego tekstu.
    \item \textbf{Rozpoznanie tekstu} - Podobnie jak \nameref{sec:paddle} docTr najpierw rozpoznaje cechy charakterystyczne tekstu przy
        pomocy CNN. Jednakże do odróżnienia znaków używana jest dwukierunkowa rekurencyjna sieć neuronowa. Sieć ta różni się od
        zwykłej sieci RNN tym że czyta sekwencje znaków od lewej do prawej a następnie od prawej do lewej. W niektórych
        przypadkach ciąg liter "cl" może być bardzo zbliżony do "d". Dlatego też zabieg ten redukuje możliwość pomylenia znaków. 
\end{enumerate}
