EasyOCR jest jednym z nowszych algorytmów użytych w tej pracy (pierwsza wersja pochodzi z 2019 roku\cite{easy2024}).
Rozwojem tego projektu zajmuje się zespół Jaided AI, specjalizujący się w wizji komputerowej i uczeniu maszynowym.
Mimo stosunkowo krótkiej historii algorytm ten stał się popularny wśród deweloperów oraz w kręgach akademickich.
Projekt ten priorytetyzuje prostotę, szybkość oraz wygodę w użytkowaniu.
Podobnie jak \nameref{sec:docTR} algorytm ten działa w dwóch krokach:
\begin{enumerate}
    \item \textbf{Wykrywanie tekstu} - W przeciwieństwie do pozostałych algorytmów EasyOCR, w tym kroku korzysta
        z modelu CRAFT (z ang. Character-Region Awareness For Text detection). Model ten tworzy dwie mapy 
        prawdopodobieństwa. W pierwszej mapie zapisane jest prawdopodobieństwo, że dany piksel znajduje się na środku
        znaku. Druga mapa zawiera prawdopodobieństwo, że dany piksel jest na środku przerwy między znakami.
        Nakładając na siebie te dwie mapy, model jest w stanie precyzyjnie wyliczyć zarys każdego znaku, a następnie 
        wyrysować wielokąt zawierający dane słowo.
    \item \textbf{Rozpoznanie tekstu} - Ten krok jest już znacznie bardziej standardowy. Przebiega podobnie jak 
        analogiczny krok w algorytmie \nameref{sec:docTR}.  
\end{enumerate}
