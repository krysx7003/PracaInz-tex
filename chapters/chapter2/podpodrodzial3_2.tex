%WER
\label{sec:WER}
WER (Word Error Rate z ang. częstotliwość błędnych słów) podobnie jak CER jest to metryka dzięki której możliwa 
jest ocena różnic między tekstem wytworzonym poprzez model OCR a tekstem rzeczywistym. Jak sama nazwa wskazuje
WER porównuje tekst na poziomie poszczególnych słów. 

\[
    WER=\frac{S+D+I}{N_w}
\]
Gdzie:
\begin{itemize}
    \item S - Liczba zamian słów (ang. Substitutions), czyli słowa które występują w obu tekstach ale te w tekście
        są różne od tych w tekście orginalnym.
    \item D - Liczba usunięć słów (ang. Deletions), czyli słowa które występują w tekście orginalnym jednakże nie ma
        ich w tekście wygenerowanym.
    \item I - Liczba wstawień słów (ang. Inserts), czyli słowa nadmiarowe których nie ma w tekście orginalnym.
    \item $N_w$ - Liczba słów w tekście (ang. Number of words)
\end{itemize}

Na przykład \linebreak

    \textbf{Tekst orginalny: }Życiem wschód, śmiercia południe; \linebreak
    \textbf{Tekst wygenerowany przez model: }Zyciem wschod, siercia poudniex; \linebreak

Aby przekstałcić tekst wygenerowany do tekstu orginalnego należy wykonać 4 zamiany(Słowa zbliżone do orginału ale nie takie same).
Więc WER=4/4=1=100\%
