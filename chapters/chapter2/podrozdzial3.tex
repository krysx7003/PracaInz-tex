CER (Character Error Rate, z ang. częstotliwość błędnych znaków) to metryka, dzięki której możliwa jest ocena
różnic między tekstem wytworzonym przez model OCR a tekstem rzeczywistym. W tym 
przypadku CER obliczane jest poprzez zsumowanie operacji(wstawień, usunięć oraz zamian znaków) potrzebnych
do uzyskania tekstu rzeczywistego. 

\[
    CER=\frac{S+D+I}{N_c}
\]
Gdzie:
\begin{itemize}
    \item S - Liczba zamian znaków (ang. Substitutions)
    \item D - Liczba usunięć znaków (ang. Deletions)
    \item I - Liczba wstawień znaków (ang. Inserts)
    \item $N_c$ - Liczba znaków w tekście (ang. Number of characters)
\end{itemize}

{\raggedright
Na przykład:

    \textbf{Tekst oryginalny: }Życiem wschód, śmiercia południe;

    \textbf{Tekst wygenerowany przez model: }Zyciem wschod, siercia poudniex;
}

Aby przekształcić tekst wygenerowany do tekstu oryginalnego, należy wykonać 4 zamiany (Brakujące znaki polskie),
1 wstawienie (Brakujące 'm' w tekście wygenerowanym) oraz 1 usunięcie ('x' nie występuje w tekście oryginalnym).
Więc $CER=6/28=0.2141\approx 21.4\%$
