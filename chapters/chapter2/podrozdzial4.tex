Ten typ generacji pseudo-losowego szumu został zaproponowany przez Kena Perlina w 1985
\cite{perlin1985}. W przeciwieństwie do czysto losowych danych, które charakteryzują się
ostrymi, nieciągłymi zmianami, szum Perlina generuje wartości zmieniające się w sposób
płynny i gradualny, co lepiej odwzorowuje procesy obserwowane w środowisku naturalnym.

%% Porównanie szumu 1d vs rand

Algorytm, choć przedstawiony na Rysunku () na przykładzie jednowymiarowym, może być 
rozszerzony do dwóch lub trzech wymiarów. Wielowymiarowe warianty szumu Perlina znajdują
zastosowanie w wielu środowiskach dla przykładu w grafice komputerowej w szczególności
podczas proceduralnej generacji tekstur. Z szumu Perlina kożysta się też w produkcji
filmów kożystających z grafiki generowanej komputerowo (CGI) oraz tworzeniu środowisk
w grach komputerowych(np. w grze Minecraft proces generacja terenu kożysta z tego 
algorytmu).

Implementacja szumu Perlina przebiega w trzech głównych etapach.

{\noindent \textbf{Definicja przestrzeni}}

    W pierwszym etapie generowana jest regularna siatka w przestrzeni $n$-wymiarowej,
    gdzie każdemu węzłowi siatki przypisywany jest losowy wektor jednostkowy, określający
    gradient w danym punkcie przestrzeni.


{\noindent \textbf{Obliczenie iloczynu skalarnego}}

    Dla dowolnego punktu $P$ w przestrzeni identyfikowana jest komórka siatki, w której
    się on znajduje. Następnie dla każdego wierzchołka $V_i$ tej komórki obliczany jest
    wektor przesunięcia $\vec{d_i} = P - V_i$ oraz iloczyn skalarny między tym wektorem a
    wektorem gradientu przypisanym do wierzchołka:
    \[
        s_i = \vec{d_i} \cdot \vec{g_i}
    \]

{\noindent \textbf{Interpolacja}}

    Ostateczna wartość szumu w punkcie $P$ obliczana jest poprzez interpolację wartości
    $s_i$ uzyskanych dla wszystkich wierzchołków komórki. Interpolacja przeprowadzana jest
    przy użyciu funkcji wygładzającej, która zapewnia ciągłość pochodnych na granicach
    komórek.   
    \vspace{0.5em}
    
    W wykorzystanej implementacji, która pochodzi z biblioteki \texttt{noise} podczas 
    generacji szumu Perlina można podać następujące parametry.

\begin{enumerate}
    \item \textbf{x,y} - Kordynaty generowanego punktu.
    \item \textbf{Liczba oktaw (Octaves)} - Określa ilość nakładanych warstw szumu.
        Większa liczba oktaw powoduje dodanie detali o wyższych częstotliwościach, co
        skutkuje bardziej złożonym i szczegółowym wzorem wyjściowym.
    \item \textbf{Trwałość (Persistence)} - Współczynnik określający amplitudę kolejnych
        oktaw. Przyjmuje wartości z zakresu [0, 1]. Wysokie wartości powodują 
        zachowanie większej ilości szczegółów, generując ostrzejsze wzory. Natomiast
        niskie wartości bardziej wygładzonych rezultatów.
    \item \textbf{Lacunarity} - Parametr określający współczynnik zmiany częstotliwości
        między kolejnymi oktawami. Wartość musi być większa od 1. Wyższe wartości
        powodują szybszy wzrost częstotliwości w kolejnych oktawach, co skutkuje
        większym zróżnicowaniem skali detali.
    \item \textbf{Repeatx} - Parametr określający powtarzalność szumu w osi X. Jeżeli
        parametr ten równa się szerokości obrazu wzór nie będzie się powtarzał.
    \item \textbf{Repeaty} - Parametr określający powtarzalność szumu w osi Y. Jeżeli
        parametr ten równa się wysokości obrazu wzór nie będzie się powtarzał.

\end{enumerate}
