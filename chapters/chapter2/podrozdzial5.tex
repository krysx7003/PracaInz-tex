\subsection{Test Wilcoxona}
    Test Wilcoxona jest dobrym zamiennikiem testu T-Studenta w przypadku, w którym
    rozkład różnic między parami obserwacji nie jest normalny. W tym teście
    zakładana jest słabsza hipoteza. Mianowicie rozkład różnic jest symetryczny
    wokół jednej wartości. Celem tego testu jest udowodnienie, że ta wartość
    znacznie różni się od zera.
    \noindent\textbf{Założenia testu:}
    \begin{enumerate}
        \item Rozkład różnic jest symetryczny wokół mediany
        \item Skala pomiarowa jest przynajmniej porządkowa
        \item Różnice między parami są niezależne
    \end{enumerate}
    
    \noindent\textbf{Procedura obliczeniowa:}
    \begin{enumerate}
        \item Obliczenie różnic $d_i$ dla każdej pary obserwacji
        \item Usunięcie różnic równych zero
        \item Uporządkowanie wartości bezwzględnych różnic od najmniejszej do największej i przypisanie im rang
        \item Sumowanie rang dla różnic dodatnich ($W^+$) i ujemnych ($W^-$)
        \item Statystyka testowa $W$ to mniejsza z sum: $W = \min(W^+, W^-)$
    \end{enumerate}
