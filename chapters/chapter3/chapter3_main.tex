\Chapter{Modyfikatory tekstu}\label{chapter:rozdzialtrzeci}
    \section{Kroje pisma}
    Jednym z problemów, na które często napotykają się algorytmy OCR, jest pismo odręczne (oraz kroje pisma je imitujące).
    Ten typ pisma jest bardzo zróżnicowany. Każdy człowiek ma swój własny charakter pisma. Zapisane znaki różnią się
    kształtem, rozmiarem czy nachyleniem. Ponadto zdarza się, że w tym samym zdaniu te same znaki mogą się znacznie
    różnić. Wszystkie te czynniki sprawiają, że wiele algorytmów OCR ma mniejszą skuteczność w rozpoznawaniu pisma
    odręcznego niż pisma maszynowego.

    W tej pracy wybrano cztery kroje pisma. Pierwsze dwa mają przypominać pismo odręczne, a pozostałe dwa są popularnymi
    przykładami standardowych krojów.
    
{\noindent \large \textbf{Allura Regular}}

Ten krój ma symulować pismo odręczne. Jednakże, jak widać na poniższej próbce 
(Rys \ref{fig:allura}), znaki w tym kroju są ze sobą połączone. Może to utrudniać
rozpoznanie poszczególnych znaków, co wpłynie na wyniki algorytmów.

\begin{figure}[H]
    \centering
    \includegraphics[width=\textwidth]{images/Allura.png}
    \caption{Próbka tekstu z użyciem kroju Allura ("Lorem ipsum dolor sit amet")}\label{fig:allura}
\end{figure}

{\noindent \large \textbf{Caveat}}

Podobnie jak poprzedni krój, Caveat ma przypominać pismo odręczne. Jednakże
w przeciwieństwie do poprzednika, w tym kroju znaki są wyraźnie rozdzielone 
(Rys \ref{fig:caveat}).

\begin{figure}[H]
    \centering
    \includegraphics[width=\textwidth]{images/Caveat.png}
    \caption{Próbka tekstu z użyciem kroju Caveat ("Lorem ipsum dolor sit amet")}\label{fig:caveat}
\end{figure}


{\noindent \large \textbf{Times New Roman}}

W przeciwieństwie do poprzednich krojów, Times New Roman nie jest stylizowany
na pismo odręczne (Rys \ref{fig:times}). Jest to jeden z bardziej popularnych
krojów szeryfowych. Szeryfy to poprzeczne lub ukośne zakończenia głównych %% Citation needed 
pociągnięć znaków.

\begin{figure}[H]
    \centering
    \includegraphics[width=\textwidth]{images/Times.png}
    \caption{Próbka tekstu z użyciem kroju Times New Roman ("Lorem ipsum dolor sit amet")}\label{fig:times}
\end{figure}

{\noindent \large \textbf{Ubuntu}}

Kroje szeryfowe są popularne w publikacjach papierowych, jednak podczas
czytania tekstu elektronicznego mogą być męczące dla oczu. Dlatego 
obecnie coraz częściej stosowane są kroje bez-szeryfowe (ang. sans-serif).
Ubuntu jest przykładem takiego kroju (Rys \ref{fig:ubuntu}).

\begin{figure}[H]
    \centering
    \includegraphics[width=\textwidth]{images/Ubuntu.png}
    \caption{}Próbka tekstu z użyciem kroju Ubuntu ("Lorem ipsum dolor sit amet")\label{fig:ubuntu}
\end{figure}


    \section{Zniekształcenia obrazu}
    Kolejnym problemem jest jakość dostarczonego obrazu. W rzeczywistości rzadko spotykane są przypadki, gdy skanowany tekst
    jest równomiernie oświetlony, nie ma na nim cieni lub papier, na którym znajduje się tekst, jest niepogięty. Takie 
    zniekształcenia mogą utrudnić rozdzielenie tekstu od tła, przez co skuteczność algorytmu zmaleje.
    
    W tej pracy wybrano trzy rodzaje takich zniekształceń obrazu. Są to:
    
{\RaggedRight \large \textbf{Przesunięcie perspektywy}}

    Obrazy poddawane są losowej deformacji przestrzennej poprzez przekształcenie projektowe. Dla każdego obrazu
    generowane są cztery współczynniki deformacji z zakresu [0.01, 0.2], które określają przemieszczenie
    narożników. Na podstawie punktów źródłowych i docelowych obliczana jest macierz transformacji, a następnie
    aplikowana na obraz wejściowy przy użyciu funkcji \texttt{cv2.warpPerspective}.

    \lstinputlisting[     
        language=Python,
        caption={Fragment autorskiej klasy ImageGenerator odpowiedzialnej za generację przekrzywionych obrazów},
        label={lst:generate_tilt}
    ]{code/generate_tilt.py} 

{\RaggedRight \large \textbf{Zacienienie}}

    Na oryginalny obraz nałożono losowo wygenerowane cienie, wygenerowane przy pomocy metody \texttt{pnoise2} z
    biblioteki \texttt{noise}, która służy do generowania losowego szumu Perlina. Dobór wartości parametrów tej
    funkcji przeprowadzono metodą eksperymentalną, poprzez wielokrotną iterację z różnymi konfiguracjami, aż do
    osiągnięcia pożądanego efektu w postaci delikatnego, lokalnego zacienienia.

    \lstinputlisting[     
        language=Python,
        caption={Fragment autorskiej klasy ImageGenerator odpowiedzialnej za generację zacienionych obrazów},
        label={lst:generate_shadow}
    ]{code/generate_shadow.py} 

    Wygenerowany w ten sposób szum został poddany normalizacji. Następnie otrzymaną mapę cieni nałożono na 
    oryginalny obraz.

{\RaggedRight \large \textbf{Efekt zmiętej kartki}}

    Oryginalny obraz został poddany deformacji, obliczonej na podstawie losowo wygenrowaneg szumu Perlina.
    W celu otrzymania tego efektu, pdobnie jak w przypadku obrazów zacienionych skorzystano z metody 
    \texttt{pnoise2} biblioteki \texttt{noise}. Także procedura wyboru parametrów była zbliżona do
    poprzednika. Jednakże w tym przypadku pożądanym efektem były podłużne, nieregularne formy imitujące naturalne
    zgięcia i deformacje materiału papieru, takie jak występują w fizycznych, użytkowanych dokumentach.

    \lstinputlisting[     
        language=Python,
        caption={Fragment autorskiej klasy ImageGenerator odpowiedzialnej za generację zmiętych obrazów},
        label={lst:generate_wrinkle}
    ]{code/generate_wrinkle.py} 

    Po wygenerowaniu otrzymany szum był normalizowany, w celu otrzymania mapy przesunięć. Następnie
    otrzymaną w ten sposób mapę nakładano na oryginalny obraz przy pomocy wbudowanych funkcji biblioteki
    \texttt{cv2}.


