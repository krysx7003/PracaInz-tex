
{\RaggedRight \large \textbf{Allura Regular}}

Ten krój ma symulować pismo odręczne. Jednakże jak widać na poniższej próbce 
(Rys \ref{fig:allura}) znaki w tym kroju są ze sobą połączone. Może to utrudniać
rozpoznanie poszczególnych znaków, co wpłynie na wyniki algorytmów.

\begin{figure}[H]
    \centering
    \includegraphics[width=\textwidth]{images/Allura.png}
    \caption{}\label{fig:allura}
\end{figure}

{\RaggedRight \large \textbf{Caveat}}

Podobnie jak poprzedni krój, Caveat ma przypominać pismo odręcznie. Jednakże
w przeciweństwie do poprzednika w tym kroju znaki są wraźnie rozdzielone 
(Rys \ref{fig:caveat}).

\begin{figure}[H]
    \centering
    \includegraphics[width=\textwidth]{images/Caveat.png}
    \caption{}\label{fig:caveat}
\end{figure}


{\RaggedRight \large \textbf{Times New Roman}}

W przeciwieństwie do poprzednich krojów Times New Roman nie jest stylizowany
na pismo odręczne (Rys \ref{fig:times}). Jest to jeden z bardziej popularnych
krojów szeryfowych. Szeryfy to poprzeczne lub ukośne zakończenia głównych %% Citation needed 
pociągnięć znaków.

\begin{figure}[H]
    \centering
    \includegraphics[width=\textwidth]{images/Times.png}
    \caption{}\label{fig:times}
\end{figure}

{\RaggedRight \large \textbf{Ubuntu}}

Kroje szeryfowe są popularne w publikacjach papierowych, jednakże podczas
czytania tekstu elektronicznego mogą być one męczące dla oczu. Dlatego 
obecnie coraz częściej stosowane są kroje bez-szeryfowe (ang. sans-sreif).
Ubuntu jest przykładem takiego kroju (Rys \ref{fig:ubuntu}).

\begin{figure}[H]
    \centering
    \includegraphics[width=\textwidth]{images/Ubuntu.png}
    \caption{}\label{fig:ubuntu}
\end{figure}
