
{\RaggedRight \large \textbf{Przesunięcie perspektywy}}

    Obrazy poddawane są losowej deformacji przestrzennej poprzez przekształcenie projektowe. Dla każdego obrazu
    generowane są cztery współczynniki deformacji z zakresu [0.01, 0.2], które określają przemieszczenie
    narożników. Na podstawie punktów źródłowych i docelowych obliczana jest macierz transformacji, a następnie
    aplikowana na obraz wejściowy przy użyciu funkcji \texttt{cv2.warpPerspective}.

    \lstinputlisting[     
        language=Python,
        caption={Fragment autorskiej klasy ImageGenerator odpowiedzialnej za generację przekrzywionych obrazów},
        label={lst:generate_tilt}
    ]{code/generate_tilt.py} 

{\RaggedRight \large \textbf{Zacienienie}}

    Na oryginalny obraz nałożono losowo wygenerowane cienie, wygenerowane przy pomocy metody \texttt{pnoise2} z
    biblioteki \texttt{noise}, która służy do generowania losowego szumu Perlina. Dobór wartości parametrów tej
    funkcji przeprowadzono metodą eksperymentalną, poprzez wielokrotną iterację z różnymi konfiguracjami, aż do
    osiągnięcia pożądanego efektu w postaci delikatnego, lokalnego zacienienia.

    \lstinputlisting[     
        language=Python,
        caption={Fragment autorskiej klasy ImageGenerator odpowiedzialnej za generację zacienionych obrazów},
        label={lst:generate_shadow}
    ]{code/generate_shadow.py} 

    Wygenerowany w ten sposób szum został poddany normalizacji. Następnie otrzymaną mapę cieni nałożono na 
    oryginalny obraz.

{\RaggedRight \large \textbf{Efekt zmiętej kartki}}

    Oryginalny obraz został poddany deformacji, obliczonej na podstawie losowo wygenrowaneg szumu Perlina.
    W celu otrzymania tego efektu, pdobnie jak w przypadku obrazów zacienionych skorzystano z metody 
    \texttt{pnoise2} biblioteki \texttt{noise}. Także procedura wyboru parametrów była zbliżona do
    poprzednika. Jednakże w tym przypadku pożądanym efektem były podłużne, nieregularne formy imitujące naturalne
    zgięcia i deformacje materiału papieru, takie jak występują w fizycznych, użytkowanych dokumentach.

    \lstinputlisting[     
        language=Python,
        caption={Fragment autorskiej klasy ImageGenerator odpowiedzialnej za generację zmiętych obrazów},
        label={lst:generate_wrinkle}
    ]{code/generate_wrinkle.py} 

    Po wygenerowaniu otrzymany szum był normalizowany, w celu otrzymania mapy przesunięć. Następnie
    otrzymaną w ten sposób mapę nakładano na oryginalny obraz przy pomocy wbudowanych funkcji biblioteki
    \texttt{cv2}.
