\Chapter{Aspekt badawczy}\label{chapter:rozdzialczwarty}

\textbf{Protokuł eksperymentalny}
W celu zapewnienia rzetelności procedury badawczej, wszystkie ekseprymenty zostały przeprowadzone przy pomocy
2-foldowej stratyfikowanej walidacji krzyżowej powtórzonej 5-krotnie. 
Otrzymane w ten sposób wyniki porównano przy pomocy testu Wilcoxona dla par obserwacji gdzie $\alpha = 0.05$.
Aby udowodnić że różnice między wynikami nie są losowe użyto połączonego testu F dla walidacji 
krzyżowej (Combined 5×2 CV F-test).

\textbf{Metryki oceny}
Skuteczność algorytmów oceniano na podstawie współczynnika błędów znakowych (Character Error Rate).
Dla oceny wydajności obliczeniowej wykorzystano średni czas przetwarzania pojedynczej strony tekstu.

\textbf{Pytania badawcze}
\begin{enumerate}
    \item Który algorytm zapewnia najlepszy kompromis między dokładnością a szybkością przetwarzania?
    \item W jakim stopniu opracowany autorski zbiór danych jest porównywalny z uznanymi benchmarkami OCR pod względem trudności i zdolności do weryfikacji podstawowej skuteczności algorytmów?
    \item W jakim stopniu rodzaj i stopień wybranych zniekształceń wpływa na wyniki algorytmów OCR?
    \item W jakim stopniu wybrane kroje pisma wypływają na dokładność działania wybranych algorytmów?
\end{enumerate}

\section{Scenariusze eksperymentów}
Dane do zbioru zostały pozyskane z serwisu WolneLektury.pl. Dostęp do tych danych możliwy
jest poprzez publiczny iterfejs programistyczny (Application Programming Interface – API),
dostępny pod adresem bazowym: https://wolnelektury.pl/api/.

Architektura API opiera się na zasadach REST (Representational State Transfer), oferując
strukturalny dostęp do zasobów poprzez logiczne endpointy. Serwis WolneLektury.pl
udostępnia wiele punktów dostępu, jednakże te najważniejsze dla tej pracy to:

\begin{itemize}
    \item /api/books/ – Wszystkie utwory
    \item /api/authors/ – Lista autorów
\end{itemize}
Domyślnie dane otrzymywane w wyniku zapytania są serializowane w formacie JSON, możliwa jest też zmiana
na format XML (poprzez dodanie parametru ?format=xml do zapytania). Jednakże w tej pracy 
wykorzystano domyślny format.

Mechanizm filtrowania i precyzyjnego wyszukiwania rekordów realizowany jest poprzez rozbudowę
ścieżki URL o odpowiednie parametry. Na przykład zapytanie pod adres \\ /api/authors/adam-mickiewicz/
zwraca informacje o autorze "adam-mickiewicz".
\begin{figure}[H]
    \centering
    \includegraphics[width=\textwidth]{images/mickiewicz.png}
    \caption{}\label{fig:mickiewicz}
\end{figure}

Najważniejszą funkcjonalnością tego API z punktu widzenia tej pracy jest możliwość łączenia 
endpointów w łańcuchy zapytań. Dzięki temu możliwe jest dokładne definiowanie podzbiorów danych.
Dla ilustracji, zapytanie /api/authors/adam-mickiewicz/books/ zwraca metadane dla książek, których
autorem jest Adam Mickiewicz.
\begin{figure}[H]
    \centering
    \includegraphics[width=\textwidth]{images/mickiewicz-books.png}
    \caption{}\label{fig:mickiewicz-books}
\end{figure}
Wynik takiego zapytania zawiera listę skróconych opisów pozycjii. Aby
otrzymać dokładniejsze dane na temat danej książki należy wykorzystać atrybut "href", który zawiera
link do sczegółowego opisu pozycji.
\begin{figure}[H]
    \centering
    \includegraphics[width=\textwidth]{images/ballady-i-romanse-1.png}
    \caption{}\label{fig:ballady-1}
\end{figure}

\begin{figure}[H]
    \centering
    \includegraphics[width=\textwidth]{images/ballady-i-romanse-2.png}
    \caption{}\label{fig:ballady-2}
\end{figure}

W sczegółowych danych na temat książki najważniejsze, z punktu widzenia tej pracy, były pola "epub",
"html","txt" oraz "xml". Zawierają one linki do pobrania zawartości książki w odpowiednim formacie.

\lstinputlisting[     
    language=Python,
    caption={Fragment kodu odpowiedzialny za pobieranie metadanych pozycji.},
    label={lst:book-list}
]{code/book_list.py}
Podczas tworzenia zbioru stała AUTHOR zawierała napis "adam-mickiewicz". Ten fragment kodu
odpowiedzialny jest za pobranie listy książek dla wybranego autora. Book to autorska
klasa służąca do deserializacji danych w formacie JSON.

\lstinputlisting[     
    language=Python,
    caption={Fragment kodu odpowiedzialny za pobranie podanej pozycji},
    label={lst:scrape}
]{code/scrape.py}


\section{Eksperymenty}
Początkowo rozważono wykorzystanie formatów tekstowych, takich jak TXT, HTML oraz XML, które
ze względu na swoją prostą, czytelną dla maszyn strukturę, wydawały się obiecującym
rozwiązaniem problemu ekstrakcji treści. Proces pozyskania danych z tych formatów sprowadzałby
się wówczas do bezpośredniego odczytu i parsowania plików.

Okazało się jednak, że zasoby udostępniane w tych formatach przez serwis WolneLektury.pl są
często niekompletne w kontekście całych zbiorów dzieł. Dla zilustrowania tego problemu można
posłużyć się przykładem cyklu „Ballady i romanse” Adama Mickiewicza, który składa się z czternastu
osobnych utworów. Tymczasem pliki w formatach TXT, HTML i XML dostępne do pobrania dla tego cyklu
zawierają wyłącznie tekst pojedynczego wiersza „To lubię”, pomijając pozostałe części dzieła.

Wobec powyższych ograniczeń, ostatecznym wyborem formatu źródłowego do budowy korpusu wybrany 
został format EPUB. Format ten można traktować jako format binarny, jednakże w rzeczywistości
jest to archiwum zawierające ustrukturyzowane pliki (X)HTML, metadane oraz zasoby multimedialne.
Taka struktura znacząco komplikuje proces ekstrakcji tekstu w porównaniu z prostymi formatami
tekstowymi. Aby uprościć proces wydobywania tekstu z plików EPUB, wykorzystano biblioteki:
\texttt{EbookLib} do niskopoziomowego parsowania struktury archiwum EPUB oraz
\texttt{Beautiful Soup} do przetwarzania i ekstrakcji czystego tekstu z sekcji HTML zawartych
w książce.

\lstinputlisting[     
    language=Python,
    caption={Fragment autorskiej klasy TextExtractor odpowiedzialnej za ekstrakcję tekstu},
    label={lst:extract}
]{code/extract.py}
Plik źródłowy zostaje otwarty przy pomocy funkcji z biblioteki \texttt{EbookLib}. Otrzymane
w ten sposób to pliki składowe archiwum jakim jest format EPUB. Pliki te składają się z 
wielu elementów, jednakże na potrzeby tej pracy wykorzystano dane z elementów typu
"h2" i "div". Zawierają one odpowiednio nagłówki oraz główny tekst z publikacji.

