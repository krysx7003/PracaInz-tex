Dane do zbioru zostały pozyskane z serwisu WolneLektury.pl. Dostęp do tych danych możliwy
jest poprzez publiczny interfejs programistyczny (Application Programming Interface – API),
dostępny pod adresem bazowym: https://wolnelektury.pl/api/.

Architektura API opiera się na zasadach REST (Representational State Transfer), oferując
strukturalny dostęp do zasobów poprzez logiczne endpointy. Serwis WolneLektury.pl
udostępnia wiele punktów dostępu. Jednak te najważniejsze dla tej pracy to:

\begin{itemize}
    \item /api/books/ – Wszystkie utwory
    \item /api/authors/ – Lista autorów
\end{itemize}
Domyślnie dane otrzymywane w wyniku zapytania są serializowane w formacie JSON, możliwa jest też zmiana
na format XML (poprzez dodanie parametru ?format=xml do zapytania). Jednakże w tej pracy 
wykorzystano domyślny format.

Mechanizm filtrowania i precyzyjnego wyszukiwania rekordów realizowany jest poprzez rozbudowę
ścieżki URL o odpowiednie parametry. Na przykład, zapytanie pod adres \\ /api/authors/adam-mickiewicz/
zwraca informacje o autorze "adam-mickiewicz".
\begin{figure}[H]
    \centering
    \includegraphics[width=\textwidth]{images/mickiewicz.png}
    \caption{Wynik zapytania /api/authors/adam-mickiewicz/}\label{fig:mickiewicz}
\end{figure}

Najważniejszą funkcjonalnością tego API z punktu widzenia tej pracy jest możliwość łączenia 
endpointów w łańcuchy zapytań. Dzięki temu możliwe jest dokładne definiowanie podzbiorów danych.
Dla ilustracji, zapytanie /api/authors/adam-mickiewicz/books/ zwraca metadane dla książek, których
autorem jest Adam Mickiewicz.
\begin{figure}[H]
    \centering
    \includegraphics[width=\textwidth]{images/mickiewicz-books.png}
    \caption{Fragment wyniku zapytania /api/authors/adam-mickiewicz/books/}\label{fig:mickiewicz-books}
\end{figure}
Wynik takiego zapytania zawiera listę skróconych opisów pozycji. Aby
otrzymać dokładniejsze dane na temat danej książki, należy wykorzystać atrybut "href", który zawiera
link do szczegółowego opisu pozycji.

\begin{figure}[H]
    \centering
    \includegraphics[width=\textwidth]{images/ballady-i-romanse-2.png}
    \caption{Widok szczegółówy dla pozycji ballady i romanse}\label{fig:ballady-2}
\end{figure}

W szczegółowych danych na temat książki najważniejsze, z punktu widzenia tej pracy, były pola "epub",
"html","txt" oraz "xml". Zawierają one linki do pobrania zawartości książki w odpowiednim formacie.

\lstinputlisting[     
    language=Python,
    caption={Fragment kodu odpowiedzialny za pobieranie metadanych pozycji.},
    label={lst:book-list}
]{code/book_list.py}
Podczas tworzenia zbioru, stała AUTHOR zawierała napis "adam-mickiewicz". Ten fragment kodu
 jest odpowiedzialny za pobranie listy książek dla wybranego autora. Book to autorska
klasa służąca do deserializacji danych w formacie JSON.

\lstinputlisting[     
    language=Python,
    caption={Fragment kodu odpowiedzialny za pobranie podanej pozycji},
    label={lst:scrape}
]{code/scrape.py}
