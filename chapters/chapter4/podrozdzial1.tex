\textbf{Eksperyment 1 \- Porównanie szybkości przetwarzania a jakości wyników }
Celem pierwszego eksperymentu jest analiza związku między czasem przetwarzania a jakością
otrzymanych wyników.

\textbf{Eksperyment 2 \- Porównanie autorskiego zbioru z uznanymi benchmarkami }
Celem drugiego eksperymentu jest uzasadnienie, autorski zbiór danych jest porównywalny
z innymi benchmarkami algorytmów OCR(Zbiory IAM oraz old-books-dataset).

\textbf{Eksperyment 3 \- Wpływ zniekształceń obrazu }
Trzeci eksperyment ma na celu zbadanie wpływu zastosowanych zniekształceń obrazu (Pochylenie 
perspektywiczne, falowanie tekstu, częściowe zacienienie, Symulacja kropli wody na obiektywie)
na otrzymane wyniki.

\textbf{Eksperyment 4 \- Wpływ krojów pisma}
Natomiast celem czwartego eksperymentu jest zbadanie wpływu wybranych krojów pisma (Brightlast,
Tangerine,Ubuntu-B,Times New Roman oraz Fira Mono) na otrzymane wyniki.
