Obrazy wyjściowe generowane były w rozdzielczości odpowiadającej standardowemu formatowi A4
(2480 × 3508 pikseli), przy założeniu rozdzielczości 300 DPI (dots per inch). W przyjętej 
konfiguracji tekst na obrazach miał rozmiar 60 pikseli, a marginesy 100 pikseli. W celu
zapewnienia pełnej widoczności tekstu oraz uniknięcia jego przycięcia, podzielono go na 
fragmenty zawierające $x$ linii tekstu. Wartość $x$ wyliczono za pomocą poniższego wzoru:
\[
    x = \left\lfloor \frac{H - 2m}{h} \right\rfloor
\]
Gdzie:
\begin{itemize}
    \item $H$ - Wysokość obrazu (3508 px)
    \item $m$ - Margines (100 px)
    \item $r$ - Rozmiar tekstu (60 px + odstęp między wierszowy)
\end{itemize}

Do generacji obrazów wykorzystano bibliotekę \texttt{Pillow}, jest to standardowa 
biblioteka do generacji obrazów w języku Python. Zaletą wykorzystania tej 
biblioteki jest prostota zmiany kroju pisma użytego do generacji. \texttt{Pillow}
udostępnia klasę FreeTypeFont, która pozwala na załadowanie pliku w formacie ttf.
Następnie załadowany krój pisma można przekazać jako argument podczas rysowania 
tekstu.

\lstinputlisting[     
    language=Python,
    caption={Fragment autorskiej klasy ImageGenerator odpowiedzialnej za generację obrazów},
    label={lst:generate_clean}
]{code/generate_clean.py} 

