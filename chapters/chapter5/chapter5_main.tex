\Chapter{Aspekt badawczy}\label{chapter:rozdzialpiaty}

{\noindent \large \textbf{Pytania badawcze}}
\begin{enumerate}
    \item W jakim stopniu opracowany autorski zbiór danych jest wykazuje porównywalność z uznanymi benchmarkami OCR pod względem
        trudności i zdolności do weryfikacji podstawowej skuteczności algorytmów?
    \item Czy zastosowane kategorie zniekształceń wizualnych wpływają w sposób statystycznie istotny na efektywność działania algorytmów OCR?
    \item Czy wybrane kroje pisma mają statystycznie istotny wpływ na efektywność działania wybranych algorytmów?
\end{enumerate}

{\noindent \large \textbf{Protokuł eksperymentalny}}

    W celu zapewnienia rzetelności procedury badawczej, wszystkie ekseprymenty zostały przeprowadzone przy pomocy
    2-foldowej walidacji krzyżowej powtórzonej 5-krotnie. 
    Otrzymane w ten sposób wyniki porównano przy pomocy testu Wilcoxona dla par obserwacji gdzie $\alpha = 0.05$.
    Aby udowodnić że różnice między wynikami nie są losowe użyto połączonego testu F dla walidacji 
    krzyżowej (Combined 5×2 CV F-test).
    \vspace{0.5em}

\section{Scenariusze eksperymentów}
{\noindent \textbf{Eksperyment 1 \- Porównanie autorskiego zbioru z uznanymi benchmarkami }}

    Celem tego eksperymentu było uzasadnienie, że autorski zbiór danych jest porównywalny
    z innymi benchmarkami algorytmów OCR(zbiory IAM oraz old-books-dataset). Eksperyment
    został przeprowadzony na czystej wersji (bez zniekształceń oraz kroju pisma Ubuntu)
    autorskiego zbioru danych. Eksperyment przeprowadzono w celu odpowiedzi na pytanie badawcze
    nr. 1. Wyniki zostały porównane na podstawie współczynnika błędów znakowych (CER). 
    \vspace{0.5em}

{\noindent \textbf{Eksperyment 2 \- Wpływ zniekształceń obrazu }}

    Trzeci eksperyment ma na celu zbadanie wpływu zastosowanych zniekształceń obrazu (opisane
    w rozdziale \ref{chapter:rozdzialtrzeci}) na uzyskane wyniki. Dla każdego rodzaju
    zniekształcenia został wygenerowany nowy fragment autorskiego zbioru danych. Eksperyment
    zostanie przeprowadzony na tych fragmentach. Eksperyment przeprowadzono w celu odpowiedzi
    na pytanie badawcze nr. 2. Wyniki zostały porównane na podstawie metryki współczynnika
    błędów znakowych (CER).
    \vspace{0.5em}

{\noindent \textbf{Eksperyment 3 \- Wpływ krojów pisma}}

    Natomiast celem czwartego eksperymentu było zbadanie wpływu wybranych krojów pisma (opisane
    w rozdziale \ref{chapter:rozdzialtrzeci}) na otrzymane wyniki. Dla każdego kroju pisma
    został wygenerowany nowy fragment autorskiego zbioru danych. Eksperyment zostanie 
    przeprowadzony na tych fragmentach. Eksperyment przeprowadzono w celu odpowiedzi na pytanie
    badawcze nr. 3. Wyniki zostały porównane na podstawie metryki współczynnika błędów
    znakowych (CER).

