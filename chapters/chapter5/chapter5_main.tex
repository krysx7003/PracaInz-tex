\Chapter{Plan eksperymentów}\label{chapter:rozdzialpiaty}

{\noindent \large \textbf{Pytania badawcze}}
\begin{enumerate}
    \item W jaki sposób autorski zbiór danych różni się od wybranych publicznych benchmarków OCR pod względem zdolności
        do różnicowania skuteczności analizowanych algorytmów?
    \item Który z badanych algorytmów OCR osiąga statystycznie istotnie lepszą efektywność dla poszczególnych krojów pisma?
    \item Który z badanych algorytmów OCR osiąga statystycznie istotnie lepszą efektywność dla poszczególnych kategorii zniekształceń wizualnych?
\end{enumerate}

{\noindent \large \textbf{Protokół eksperymentalny}}

    W celu zapewnienia rzetelności procedury badawczej, otrzymane wyniki poddano wielokrotnej analizie statystycznej.
    Każdy z eksperymentów zostały powtórzony pięciokrotnie, co pozwoliło na zgromadzenie stabilnych oszacowań metryk
    efektywności. Tak otrzymane wyniki poddano analizie statystycznej przeprowadzonej w trzech etapach:

    \begin{enumerate}
        \item \textbf{Test Friedmana} - Wstępna weryfikacja globalnej hipotezy o istnieniu różnic między porównywanymi
            algorytmami OCR.
        \item \textbf{Test Wilcoxona dla par} - Szczegółowe porównania parami.
        \item \textbf{Miara Cliffa Delta} - Ocena wielkości efektu dla istotnych statystycznie różnic.
    \end{enumerate}

    Analizę przeprowadzono z poziomen istotności $\alpha$ równym $0.05$. Implementację protokołu eksperymentalnego
    wykonano w języku Python z wykorzystaniem funkcji z biblioteki \texttt{SciPy}.
    \vspace{0.5em}

\section{Scenariusze eksperymentów}
{\noindent \textbf{Eksperyment 1 \- Porównanie autorskiego zbioru z uznanymi benchmarkami }}

    Celem tego eksperymentu było uzasadnienie, że autorski zbiór danych jest porównywalny
    z innymi benchmarkami algorytmów OCR(zbiory IAM oraz old-books-dataset). Eksperyment
    został przeprowadzony na czystej wersji (bez zniekształceń oraz kroju pisma Ubuntu)
    autorskiego zbioru danych. Eksperyment przeprowadzono w celu odpowiedzi na pytanie badawcze
    nr. 1. Wyniki zostały porównane na podstawie współczynnika błędów znakowych (CER). 
    \vspace{0.5em}

{\noindent \textbf{Eksperyment 2 \- Wpływ zniekształceń obrazu }}

    Trzeci eksperyment ma na celu zbadanie wpływu zastosowanych zniekształceń obrazu (opisane
    w rozdziale \ref{chapter:rozdzialtrzeci}) na uzyskane wyniki. Dla każdego rodzaju
    zniekształcenia został wygenerowany nowy fragment autorskiego zbioru danych. Eksperyment
    zostanie przeprowadzony na tych fragmentach. Eksperyment przeprowadzono w celu odpowiedzi
    na pytanie badawcze nr. 2. Wyniki zostały porównane na podstawie metryki współczynnika
    błędów znakowych (CER).
    \vspace{0.5em}

{\noindent \textbf{Eksperyment 3 \- Wpływ krojów pisma}}

    Natomiast celem czwartego eksperymentu było zbadanie wpływu wybranych krojów pisma (opisane
    w rozdziale \ref{chapter:rozdzialtrzeci}) na otrzymane wyniki. Dla każdego kroju pisma
    został wygenerowany nowy fragment autorskiego zbioru danych. Eksperyment zostanie 
    przeprowadzony na tych fragmentach. Eksperyment przeprowadzono w celu odpowiedzi na pytanie
    badawcze nr. 3. Wyniki zostały porównane na podstawie metryki współczynnika błędów
    znakowych (CER).

