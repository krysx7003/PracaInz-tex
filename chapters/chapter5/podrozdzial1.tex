\textbf{Eksperyment 1 \- Porównanie szybkości przetwarzania a jakości wyników }
Celem pierwszego eksperymentu jest analiza związku między czasem przetwarzania a jakością
otrzymanych wyników. 
Eksperyment zostanie przeprowadzony na główny podzbiorze(Bez zniekształceń oraz standardowy 
krój pisma) autorskiego zbioru danych.

\textbf{Eksperyment 2 \- Porównanie autorskiego zbioru z uznanymi benchmarkami }
Celem tego eksperymentu jest uzasadnienie, autorski zbiór danych jest porównywalny
z innymi benchmarkami algorytmów OCR(Zbiory IAM oraz old-books-dataset).
Podobnie jak w przypadku eksperymentu 1 w tym ekseprymencie zostanie wykorzystany główny 
podozbiór autorskiego zbioru.

\textbf{Eksperyment 3 \- Wpływ zniekształceń obrazu }
Trzeci eksperyment ma na celu zbadanie wpływu zastosowanych zniekształceń obrazu (Pochylenie 
perspektywiczne, falowanie tekstu, częściowe zacienienie, Symulacja kropli wody na obiektywie)
na otrzymane wyniki.
Dla każdego rodzaju zniekształcenia zostanie wygenerowany podzbiór.
Eksperyment zostanie przeprowadzony na tych podzbiorach.

\textbf{Eksperyment 4 \- Wpływ krojów pisma}
Natomiast celem czwartego eksperymentu jest zbadanie wpływu wybranych krojów pisma (Brightlast,
Tangerine,Ubuntu-B,Times New Roman oraz Fira Mono) na otrzymane wyniki.
Dla każdego kroju pisma zostanie wygenerowany podzbiór.
Eksperyment zostanie przeprowadzony na tych podzbiorach.
