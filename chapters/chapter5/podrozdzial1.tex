{\noindent \textbf{Eksperyment 1 \- Porównanie szybkości przetwarzania i jakości wyników }}

    Celem pierwszego eksperymentu była analiza związku między czasem przetwarzania a jakością
    otrzymanych wyników. Eksperyment został przeprowadzony na czystej wersji (bez
    zniekształceń oraz krój pisma Ubuntu) autorskiego zbioru danych. Eksperyment 
    przeprowadzono w celu odpowiedzi na pytanie badawcze nr. 1.
    \vspace{0.5em}

{\noindent \textbf{Eksperyment 2 \- Porównanie autorskiego zbioru z uznanymi benchmarkami }}

    Celem tego eksperymentu było uzasadnienie, autorski zbiór danych jest porównywalny
    z innymi benchmarkami algorytmów OCR(zbiory IAM oraz old-books-dataset). Podobnie jak
    w przypadku eksperymentu 1 w tym ekseprymencie została wykorzystana czysta wersja
    autorskiego zbioru. Eksperyment przeprowadzono w celu odpowiedzi na pytanie badawcze
    nr. 2. Wyniki zostały porównywane na podstawie współczynnika błędów znakowych (CER). 
    \vspace{0.5em}

{\noindent \textbf{Eksperyment 3 \- Wpływ zniekształceń obrazu }}

    Trzeci eksperyment ma na celu zbadanie wpływu zastosowanych zniekształceń obrazu (opisane
    w rozdziale \ref{chapter:rozdzialtrzeci}) na otrzymane wyniki. Dla każdego rodzaju
    zniekształcenia został wygnenerowany nowy fragment autorskiego zbioru danych. Eksperyment
    zostanie przeprowadzony na tych fragmentach. Eksperyment przeprowadzono w celu odpowiedzi
    na pytanie badawcze nr. 3. Wyniki zostały porównywane na podstawie metryki współczynnika
    błędów znakowych (CER).
    \vspace{0.5em}

{\noindent \textbf{Eksperyment 4 \- Wpływ krojów pisma}}

    Natomiast celem czwartego eksperymentu było zbadanie wpływu wybranych krojów pisma (opisane
    w rozdziale \ref{chapter:rozdzialtrzeci}) na otrzymane wyniki. Dla każdego kroju pisma
    został wygnenerowany nowy fragment autorskiego zbioru danych. Eksperyment zostanie 
    przeprowadzony na tych fragmentach. Eksperyment przeprowadzono w celu odpowiedzi na pytanie
    badawcze nr. 4. Wyniki zostały porównywane na podstawie metryki współczynnika błędów
    znakowych (CER).
