\Chapter{Wyniki}\label{chapter:rozdzialszosty}

\section{Eksperyment 1 Porówanie z innymi zbiorami}

    \begin{table}[!ht]
\centering
\caption{Wyniki algorytmów OCR dla zbioru korzystającego z kroju Ubuntu}
\label{tab:clean_repeat_summary}
\begin{tabular}{lrrrr}
\toprule
 & Średnie CER & \begin{tabular}[c]{@{}l@{}}Odchylenie\\ standardowe dla CER\end{tabular} & Średni czas[s] & \begin{tabular}[c]{@{}l@{}}Odchylenie\\ standardowe dla czasu\end{tabular} \\
\midrule
DocTr & 0.0972 & 0.0000 & 2.9361 & 0.1956 \\
Easyocr & 0.1795 & 0.0000 & 1.8550 & 0.0313 \\
Paddle & 0.0692 & 0.0000 & 1.2890 & 0.0102 \\
Tesseract & 0.0761 & 0.0000 & 1.3744 & 0.0139 \\
\bottomrule
\end{tabular}
\end{table}

    \begin{table}[!ht]
\centering
\caption{Szczegółowe wyniki dla każdego algorytmu}
\label{tab:exp1_results}
\begin{tabular}{lrrrrrrrr}
\toprule
 & \multicolumn{2}{c}{DocTr} & \multicolumn{2}{c}{Easyocr} & \multicolumn{2}{c}{Paddle} & \multicolumn{2}{c}{Tesseract} \\ \hline
Powtórzenie & CER & Czas[s] & CER & Czas[s] & CER & Czas[s] & CER & Czas[s] \\
\midrule
1 & 0.0972 & 2.7302 & 0.1795 & 1.8428 & 0.0692 & 1.2990 & 0.0761 & 1.3611 \\
2 & 0.0972 & 2.7207 & 0.1795 & 1.8059 & 0.0692 & 1.3013 & 0.0761 & 1.3582 \\
3 & 0.0972 & 3.0820 & 0.1795 & 1.8829 & 0.0692 & 1.2821 & 0.0761 & 1.3843 \\
4 & 0.0972 & 3.1245 & 0.1795 & 1.8673 & 0.0692 & 1.2816 & 0.0761 & 1.3794 \\
5 & 0.0972 & 3.0231 & 0.1795 & 1.8758 & 0.0692 & 1.2809 & 0.0761 & 1.3888 \\
\bottomrule
\end{tabular}
\end{table}

    
    \begin{table}[!ht]
\centering
\caption{Wyniki algorytmów OCR dla zbioru korzystającego z kroju Ubuntu}
\label{tab:clean_summary}
\begin{tabular}{lrrrr}
\toprule
 & Średnie CER & \begin{tabular}[c]{@{}l@{}}Odchylenie\\ standardowe dla CER\end{tabular} & Średni czas[s] & \begin{tabular}[c]{@{}l@{}}Odchylenie\\ standardowe dla czasu\end{tabular} \\
\midrule
DocTr & 0.0972 & 0.0153 & 2.7302 & 0.3807 \\
Easyocr & 0.1795 & 0.0461 & 1.8428 & 0.2465 \\
Paddle & 0.0692 & 0.0118 & 1.2990 & 0.0903 \\
Tesseract & 0.0761 & 0.0117 & 1.3611 & 0.1767 \\
\bottomrule
\end{tabular}
\end{table}


    Jako pierwszy, wykonano test dla wersji zbioru autorskiego, korzystającej z kroju
    Ubuntu. Tabela \ref{tab:clean_repeat_summary} zawiera uśrednione wyniki pomiędzy pięcioma
    powtórzeniami. Jednakże jak widać odchylenie standardowe dla CER, kluczowej metryki
    której użyto do oceny wydajności algorytmów, jest w każdym przypadku równe 0.
    Po dokładniejszej analizie (Tabela \ref{tab:exp1_results}) ustalono że taki wynik
    jest poprawny. Dlatego też kolejne testy były analizowane na podstawie pierwszego
    powtórzenia.

    \begin{table}[!ht]
\centering
\caption{Wyniki algorytmów OCR dla zbioru korzystającego z kroju Ubuntu}
\label{tab:clean_summary}
\begin{tabular}{lrrrr}
\toprule
 & Średnie CER & \begin{tabular}[c]{@{}l@{}}Odchylenie\\ standardowe dla CER\end{tabular} & Średni czas[s] & \begin{tabular}[c]{@{}l@{}}Odchylenie\\ standardowe dla czasu\end{tabular} \\
\midrule
DocTr & 0.0972 & 0.0153 & 2.7302 & 0.3807 \\
Easyocr & 0.1795 & 0.0461 & 1.8428 & 0.2465 \\
Paddle & 0.0692 & 0.0118 & 1.2990 & 0.0903 \\
Tesseract & 0.0761 & 0.0117 & 1.3611 & 0.1767 \\
\bottomrule
\end{tabular}
\end{table}
 % IAM placeholder
    \begin{table}[!ht]
\centering
\caption{Wyniki algorytmów OCR dla zbioru korzystającego z kroju Ubuntu}
\label{tab:clean_summary}
\begin{tabular}{lrrrr}
\toprule
 & Średnie CER & \begin{tabular}[c]{@{}l@{}}Odchylenie\\ standardowe dla CER\end{tabular} & Średni czas[s] & \begin{tabular}[c]{@{}l@{}}Odchylenie\\ standardowe dla czasu\end{tabular} \\
\midrule
DocTr & 0.0972 & 0.0153 & 2.7302 & 0.3807 \\
Easyocr & 0.1795 & 0.0461 & 1.8428 & 0.2465 \\
Paddle & 0.0692 & 0.0118 & 1.2990 & 0.0903 \\
Tesseract & 0.0761 & 0.0117 & 1.3611 & 0.1767 \\
\bottomrule
\end{tabular}
\end{table}
 % obd placeholder

\section{Eksperyment 2 Wpływ zniekształceń obrazu}

    Wyniki działania algorytmów dostępne są w tabeli \ref{tab:clean_summary}. Z tego
    powodu nie zostały ponownie umieszczone w tej sekcji.
    \begin{table}[!ht]
\centering
\caption{Test Wilcoxona dla wyników testu na zbiorze korzystającym z kroju Ubuntu}
\label{tab:clean_vs}
\begin{tabular}{lrrr}
\toprule
 & W & Wartość p & Delta Cliffa \\
\midrule
Paddle vs Tesseract & 1856.5000 & 0.0000 & -0.3397 \\
Paddle vs DocTR & 0.0000 & 0.0000 & -0.8606 \\
Paddle vs EasyOCR & 0.0000 & 0.0000 & -0.9961 \\
Tesseract vs DocTR & 15.5000 & 0.0000 & -0.7423 \\
Tesseract vs EasyOCR & 0.0000 & 0.0000 & -0.9933 \\
DocTR vs EasyOCR & 0.0000 & 0.0000 & -0.9584 \\
\bottomrule
\end{tabular}
\end{table}


    \begin{table}[!ht]
\centering
\caption{Wyniki algorytmów OCR dla zbioru korzystającego z kroju Allura}
\label{tab:hand_connected_summary}
\begin{tabular}{lrrrr}
\toprule
 & Średnie CER & \begin{tabular}[c]{@{}l@{}}Odchylenie\\ standardowe dla CER\end{tabular} & Średni czas[s] & \begin{tabular}[c]{@{}l@{}}Odchylenie\\ standardowe dla czasu\end{tabular} \\
\midrule
DocTr & 0.2108 & 0.0245 & 2.5894 & 0.3481 \\
Easyocr & 0.4348 & 0.0568 & 1.9503 & 0.2775 \\
Paddle & 0.0775 & 0.0113 & 1.2232 & 0.0756 \\
Tesseract & 0.3199 & 0.0584 & 1.2542 & 0.1543 \\
\bottomrule
\end{tabular}
\end{table}

    \vspace{2em}
    \begin{table}[!ht]
\centering
\caption{Test Wilcoxona dla wyników testu na zbiorze korzystającym z kroju Allura}
\label{tab:hand_connected_vs}
\begin{tabular}{lrrr}
\toprule
 & W & Wartość p & Delta Cliffa \\
\midrule
Paddle vs Tesseract & 0.0000 & 0.0000 & -1.0000 \\
Paddle vs DocTR & 0.0000 & 0.0000 & -1.0000 \\
Paddle vs EasyOCR & 0.0000 & 0.0000 & -1.0000 \\
Tesseract vs DocTR & 45.0000 & 0.0000 & 0.9660 \\
Tesseract vs EasyOCR & 1147.0000 & 0.0000 & -0.8428 \\
DocTR vs EasyOCR & 1.0000 & 0.0000 & -0.9999 \\
\bottomrule
\end{tabular}
\end{table}


    \begin{table}[!ht]
\centering
\caption{Wyniki algorytmów OCR dla zbioru korzystającego z kroju Caveat}
\label{tab:hand_separate_summary}
\begin{tabular}{lrrrr}
\toprule
 & Średnie CER & \begin{tabular}[c]{@{}l@{}}Odchylenie\\ standardowe dla CER\end{tabular} & Średni czas[s] & \begin{tabular}[c]{@{}l@{}}Odchylenie\\ standardowe dla czasu\end{tabular} \\
\midrule
DocTr & 0.1613 & 0.0274 & 3.2743 & 0.7884 \\
Easyocr & 0.3286 & 0.0588 & 2.1997 & 0.3805 \\
Paddle & 0.1496 & 0.0246 & 1.1880 & 0.0703 \\
Tesseract & 0.1707 & 0.0252 & 1.2508 & 0.1403 \\
\bottomrule
\end{tabular}
\end{table}

    \vspace{2em}
    \begin{table}[!ht]
\centering
\caption{Test Wilcoxona dla wyników testu na zbiorze korzystającym z kroju Caveat}
\label{tab:hand_separate_vs}
\begin{tabular}{lrrr}
\toprule
 & W & Wartość p & Delta Cliffa \\
\midrule
Paddle vs Tesseract & 942.5000 & 0.0000 & -0.5044 \\
Paddle vs DocTR & 1436.5000 & 0.0000 & -0.2782 \\
Paddle vs EasyOCR & 0.0000 & 0.0000 & -0.9982 \\
Tesseract vs DocTR & 4138.5000 & 0.0000 & 0.2480 \\
Tesseract vs EasyOCR & 0.0000 & 0.0000 & -0.9912 \\
DocTR vs EasyOCR & 0.0000 & 0.0000 & -0.9933 \\
\bottomrule
\end{tabular}
\end{table}


    \begin{table}[!ht]
\centering
\caption{Wyniki algorytmów OCR dla zbioru korzystającego z kroju Times New Roman}
\label{tab:times_summary}
\begin{tabular}{lrrrr}
\toprule
 & Średnie CER & \begin{tabular}[c]{@{}l@{}}Odchylenie\\ standardowe dla CER\end{tabular} & Średni czas[s] & \begin{tabular}[c]{@{}l@{}}Odchylenie\\ standardowe dla czasu\end{tabular} \\
\midrule
DocTr & 0.0897 & 0.0140 & 2.7539 & 0.3784 \\
Easyocr & 0.1886 & 0.0460 & 1.8497 & 0.3642 \\
Paddle & 0.0672 & 0.0128 & 1.2525 & 0.1413 \\
Tesseract & 0.0787 & 0.0118 & 1.3371 & 0.1711 \\
\bottomrule
\end{tabular}
\end{table}

    \vspace{2em}
    \begin{table}[!ht]
\centering
\caption{Test Wilcoxona dla wyników testu na zbiorze korzystającym z kroju Times New Roman}
\label{tab:times_vs}
\begin{tabular}{lrrr}
\toprule
 & W & Wartość p & Delta Cliffa \\
\midrule
Paddle vs Tesseract & 1461.0000 & 0.0000 & -0.5260 \\
Paddle vs DocTR & 70.0000 & 0.0000 & -0.7737 \\
Paddle vs EasyOCR & 0.0000 & 0.0000 & -0.9996 \\
Tesseract vs DocTR & 712.5000 & 0.0000 & -0.4702 \\
Tesseract vs EasyOCR & 0.0000 & 0.0000 & -0.9992 \\
DocTR vs EasyOCR & 0.0000 & 0.0000 & -0.9915 \\
\bottomrule
\end{tabular}
\end{table}


    Warto zaznaczyć, że CER jest miarą błędu więc malejące wartości są lepsze.
    Z tego też powodu ujemna wartość Delta np. w pierwszym wierszu tabeli 
    \ref{tab:clean_vs} oznacza, że algorytm Paddle usyskał wynik lepszy od algorytmu
    Tesseract w stopniu statystycznie znaczącym.    

    W przeprowadzonym eksperymencie bardzo wyróżnia się algorytm Paddle. We
    wszystkich testch dominuje nad konkurencją. Sczególnie widoczne jest
    to dla badania przerowadzonego na zbiorze wykonanym przy użyciu kroju
    Allura. Gdzie algorytm ten uzyskał średnie CER na poziomie 0.0775 następny
    w kolejności algorytm (DocTr) uzyskał wynik na poziomie 0.2108. 

\section{Eksperyment 3 Wpływ krojów pisma}

    \begin{table}[!ht]
\centering
\caption{Wyniki algorytmów OCR dla zacienionych obrazów}
\label{tab:shadow_summary}
\begin{tabular}{lrrrr}
\toprule
 & Średnie CER & \begin{tabular}[c]{@{}l@{}}Odchylenie\\ standardowe dla CER\end{tabular} & Średni czas[s] & \begin{tabular}[c]{@{}l@{}}Odchylenie\\ standardowe dla czasu\end{tabular} \\
\midrule
DocTr & 0.1011 & 0.0172 & 2.6866 & 0.3602 \\
Easyocr & 0.1831 & 0.0477 & 1.8281 & 0.2999 \\
Paddle & 0.7344 & 0.3241 & 1.2343 & 0.2159 \\
Tesseract & 0.7219 & 0.3406 & 1.1655 & 0.2651 \\
\bottomrule
\end{tabular}
\end{table}

    \begin{table}[!ht]
\centering
\caption{Test Wilcoxona dla wyników testu na zbiorze zacienionych obrazów}
\label{tab:shadow_vs}
\begin{tabular}{lrrr}
\toprule
 & W & Wartość p & Delta Cliffa \\
\midrule
Paddle vs Tesseract & 6597.0000 & 0.6836 & -0.0086 \\
Paddle vs DocTR & 51.0000 & 0.0000 & 0.9483 \\
Paddle vs EasyOCR & 531.0000 & 0.0000 & 0.8037 \\
Tesseract vs DocTR & 58.0000 & 0.0000 & 0.9374 \\
Tesseract vs EasyOCR & 562.0000 & 0.0000 & 0.7977 \\
DocTR vs EasyOCR & 1.0000 & 0.0000 & -0.9477 \\
\bottomrule
\end{tabular}
\end{table}


    \begin{table}[!ht]
\centering
\caption{Wyniki algorytmów OCR dla przechylonych obrazów}
\label{tab:tilt_summary}
\begin{tabular}{lrrrr}
\toprule
 & Średnie CER & \begin{tabular}[c]{@{}l@{}}Odchylenie\\ standardowe dla CER\end{tabular} & Średni czas[s] & \begin{tabular}[c]{@{}l@{}}Odchylenie\\ standardowe dla czasu\end{tabular} \\
\midrule
DocTr & 0.2420 & 0.1578 & 2.7522 & 0.6201 \\
Easyocr & 0.3016 & 0.1566 & 1.8904 & 0.2970 \\
Paddle & 0.2501 & 0.3644 & 1.1734 & 0.1518 \\
Tesseract & 0.3364 & 0.3262 & 1.4357 & 0.1920 \\
\bottomrule
\end{tabular}
\end{table}

    \begin{table}[!ht]
\centering
\caption{Test Wilcoxona dla wyników testu na zbiorze przekrzywionych obrazów}
\label{tab:tilt_vs}
\begin{tabular}{lrrr}
\toprule
 & W & Wartość p & Delta Cliffa \\
\midrule
Paddle vs Tesseract & 7632.5000 & 0.0000 & -0.4583 \\
Paddle vs DocTR & 9571.5000 & 0.0002 & -0.5525 \\
Paddle vs EasyOCR & 8966.0000 & 0.0000 & -0.6000 \\
Tesseract vs DocTR & 9745.0000 & 0.0005 & -0.0348 \\
Tesseract vs EasyOCR & 13230.0000 & 0.8688 & -0.2114 \\
DocTR vs EasyOCR & 4957.0000 & 0.0000 & -0.2874 \\
\bottomrule
\end{tabular}
\end{table}


    \begin{table}[!ht]
\centering
\caption{Wyniki algorytmów OCR dla zmiętych obrazów}
\label{tab:wrinkle_summary}
\begin{tabular}{lrrrr}
\toprule
 & Średnie CER & \begin{tabular}[c]{@{}l@{}}Odchylenie\\ standardowe dla CER\end{tabular} & Średni czas[s] & \begin{tabular}[c]{@{}l@{}}Odchylenie\\ standardowe dla czasu\end{tabular} \\
\midrule
DocTr & 0.0996 & 0.0187 & 2.7331 & 0.3663 \\
Easyocr & 0.2812 & 0.0584 & 1.8696 & 0.2535 \\
Paddle & 0.2948 & 0.2081 & 1.1853 & 0.0603 \\
Tesseract & 0.1261 & 0.0427 & 1.5180 & 0.1989 \\
\bottomrule
\end{tabular}
\end{table}

    \begin{table}[!ht]
\centering
\caption{Test Wilcoxona dla wyników testu na zbiorze zmiętych obrazów}
\label{tab:wrinkle_vs}
\begin{tabular}{lrrr}
\toprule
 & W & Wartość p & Delta Cliffa \\
\midrule
Paddle vs Tesseract & 2059.0000 & 0.0000 & 0.5303 \\
Paddle vs DocTR & 1144.5000 & 0.0000 & 0.6876 \\
Paddle vs EasyOCR & 12453.5000 & 0.3529 & -0.1165 \\
Tesseract vs DocTR & 3218.5000 & 0.0000 & 0.5247 \\
Tesseract vs EasyOCR & 120.0000 & 0.0000 & -0.9642 \\
DocTR vs EasyOCR & 0.0000 & 0.0000 & -0.9984 \\
\bottomrule
\end{tabular}
\end{table}


    W eksperymencie 2 algorytm Paddle miały znaczącą przewagę. Jednakże w tym
    eksperymencie ta przewaga nie była tak znacząca. W przypadku testu dla obrazów 
    zacienionych (tabela \ref{tab:shadow_summary}) wyniki tego algorytmu były
    najgorsze. Co ciekawe algorytm DocTr w tym samym teście otrzymał najlepszy wynik.

    Ciekawym wynikiem jest też test dla tekstów przekrzywionych. W tym teście
    wszystkie algorytme otrzymały wynik CER bilski 0.25 (lub wyższy).

