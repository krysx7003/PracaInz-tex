\Chapter{Podsumowanie}\label{chapter:conclusions}

{\noindent \large \textbf{Ponawiając pytania badawcze:}}
\begin{enumerate}
    \item W jakim stopniu opracowany autorski zbiór danych wykazuje porównywalność z uznanymi benchmarkami OCR pod względem
        trudności i zdolności do weryfikacji podstawowej skuteczności algorytmów?
    \item Który z badanych algorytmów OCR osiąga statystycznie istotnie lepszą efektywność dla poszczególnych krojów pisma?
    \item Który z badanych algorytmów OCR osiąga statystycznie istotnie lepszą efektywność dla poszczególnych kategorii zniekształceń wizualnych?
\end{enumerate}

{\noindent \large \textbf{Pytanie 1}}

{\noindent \large \textbf{Pytanie 2}}

    Dla dokumentów korzystających z wybranych krojów pisma (Allura, Caveat, Ubuntu oraz Times New Roman)
    najlepszą efektywność osiąga algorytm Paddle. Sczególnie dobrze widać to w przypadku w którym
    skorzystano z kroju Allura (tabela \ref{tab:hand_connected_summary}), gdzie algorytm ten otrzymał
    średnie CER na poziomie 0.0775. Jednoznacznie dominując nad konkurencją, drugi w kolejności był
    algorytm DocTR z wynikiem 0.2108.

    Obserwowana znaczna różnica między otrzymanymi wynikami była wynikiem oczekiwanym. Algorytm Paddle OCR
    jako jednyny z wybranych umożliwiał wykożystanie akceleracji przy pomocy procesora graficznego(GPU). Ta
    funkcjonalność zapewniła mu dostęp do znacznie większej mocy obliczeniowej, co bezpośrednio przełożyło
    się na krótszy czas przetwarzania przy zachowaniu wysokiej dokładności rozpoznawania.

    Wyniki algorytmu EasyOCR były niepokojące w niemal wszystkich przypadkach były one istotnie wyższe od
    pozostałych badanych rozwiązań. Może to wskazywać na błąd w implementacji.
    \vspace{0.5em}

{\noindent \large \textbf{Pytanie 3}}

    Dla dokumentów poddanych zniekształceniom wizualnym (zacienienie, przechylenie perspektywy oraz efekt zmiętej
    kartki) najlepszą efektywność osiąga algorytm DocTR. W przeciwieństwie do poprzedniego eksperymentu ta
    dominacja nie była tak jednoznaczna. W przypadku obrazów zacienionch DocTR otrzymał wyniki istotnie
    niższe od pozostałych(tabela \ref{tab:shadow_summary}). W pozostałych przypadkach różnica między algorytmem
    DocTR a pozostałymi nie były aż tak widoczne (np. w tabeli \ref{tab:tilt_summary} różnica między DocTR a Paddle
    to jedynie 0.01).

    Obserwowana wyższość algorytmu DocTR nad pozostałymi była, w tym przypadku, wynikiem oczekiwanym. DocTR
    jest algotymem trenowanym do rozpoznawania obrazów bliższych rzeczywistym. Dla przykładu zbiór treningowy
    zawiera obrazy przedstawiające zmięte paragony lub inne dokumenty.

    Wyniki dla algorytmów Paddle oraz Tesseract otrzymane w wyniku badania na zacienionych obrazach były istotnie
    niższe od wszystkich pozostałych. Dane otrzymane w wyniku pierwszego powtórzenia porównano z pozostałymi 
    czteroma powtórzeniami. W wyniku tego działania potwierdzono, że dane otrzymane w kolejnych powtórzeniach
    się nie różnią. Przez co można wykluczyć możliwość tego, że otrzymane wyniki są anomalią. Może to jednak
    oznaczać, że efektywność tych algorytmów jest znacznie bardziej zależna od kontrastu między tekstem a 
    tłem. Jednakże potwierdzenie tej teorii wymagać będzie dalszych badań.
    
    W tym eksperymencie algorytm EasyOCR otrzymał wyniki znacznie bliższe pozostałym. Co zmniejsza ryzyko błędnej 
    implementacji.

Warte odontowania są także średnie czasy analizy obrazu. Algorytm Paddle OCR uzyskiwał istotnie niższe wyniki 
w wielu kategoriach podczas gdy czas wykonania pozostawał najniższy. Jednakże wynik ten jest zależny od zastosowania
przyspieszenia przy pomocy procesora graficznego (wykonano jeden test w którym użyto Paddle bez przyspieszenia 
,jednakże średni czas wykonania przekraczał 70 sekund). Drugim pod względem szybkości wykonywania był algorytm
Tesseract, otrzymywał on wyniki bliższe algorytmowi Paddle podczas gdy czas pozostawał zbliżony do drugiego
algortmu. Jednakże w przeciwieństwie do algorytmu Paddle Tesseract nie potrzebuje przyspieszenia GPU, więc
może być bardziej użyteczny w środowiskach w których procesor graficzny jest nie dostępny.

